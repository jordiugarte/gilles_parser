\chapter{Testing}
    To test the lexical analyzer, we have provided a set of test files in the \texttt{test} folder. These test files contain GILLES programs that cover various aspects of the language.
	We've tried to incorporate different kinds of edge cases and code structures to ensure that the lexical analyzer can handle a wide range of inputs.
	\section{Test Files}
	\begin{itemize}
		\item \texttt{Euclid.gls} - A simple program to calculate the greatest common divisor of two numbers.
		\item \texttt{InvalidSymbolEuclid.gls} - Same as above but with an invalid symbol, .
		\item \texttt{Sum.gls} - A simple program to calculate the sum of two numbers.
		\item \texttt{ThreeLoopGibberish.gls} - An unncessarily complex program to test the lexer.
		\item \texttt{InvalidAssignment.gls} - A program with invalid syntax to test error handling.
		\item \texttt{ComplexAssignment.gls} - A program with complex arithmetic expression.
		\item \texttt{Fibonacci.gls} - A program to calculate the factorial of a number.
		\item \texttt{Whitespace.gls} - Random whitespace to test whitespace handling.
	\end{itemize}

	\section{Running the Tests}
	To run the tests, we can use the following command, output will be displayed on the console:
	\begin{verbatim}
		$ make test TEST_FILE=test/Sum.gls
    \end{verbatim}
  
    \begin{minted}[fontsize=\footnotesize, linenos, frame=lines]{java}
		token: LET        	lexical unit: LET
		token: sum        	lexical unit: PROGNAME
		token: BE         	lexical unit: BE
		token: IN         	lexical unit: INPUT
		token: (          	lexical unit: LPAREN
		token: a          	lexical unit: VARNAME
		token: )          	lexical unit: RPAREN
		token: :          	lexical unit: COLUMN
		token: IN         	lexical unit: INPUT
		token: (          	lexical unit: LPAREN
		token: b          	lexical unit: VARNAME
		token: )          	lexical unit: RPAREN
		token: :          	lexical unit: COLUMN
		token: c          	lexical unit: VARNAME
		token: =          	lexical unit: ASSIGN
		token: a          	lexical unit: VARNAME
		token: +          	lexical unit: PLUS
		token: b          	lexical unit: VARNAME
		token: :          	lexical unit: COLUMN
		token: END        	lexical unit: END
		token: :          	lexical unit: COLUMN
		token: OUT        	lexical unit: OUTPUT
		token: (          	lexical unit: LPAREN
		token: c          	lexical unit: VARNAME
		token: )          	lexical unit: RPAREN
		token: :          	lexical unit: COLUMN
		token: END        	lexical unit: END

		Variables
		a	4
		b	5
		c	6
	\end{minted}

    \begin{table}[h]
		\centering
		\caption{Test datasheet}
	\end{table}