\documentclass[12pt,a4paper]{report}
\usepackage{graphicx}       % For including images
\usepackage{amsmath}        % For math symbols
\usepackage{amssymb}        % For more math symbols
\usepackage{hyperref}       % For hyperlinks in the document
\usepackage{geometry}       % For adjusting margins
\geometry{left=3cm,right=3cm,top=2.5cm,bottom=2.5cm}
\usepackage{fancyhdr}       % For custom headers and footers
\usepackage{setspace}       % For line spacing

% Custom header and footer
\pagestyle{fancy}
\fancyhf{}
\fancyhead[L]{University Report}
\fancyhead[R]{\thepage}

\begin{document}
	
	% Title Page
	\begin{titlepage}
		\centering
		\vspace*{2cm}
		
		\Huge
		\textbf{Lexer Compiler}
		
		\vspace{1.5cm}
		
		\Large
		\textbf{Sahay Siddharth} \\
		\textbf{Ugarte Jordi}		
		
		\vfill
		
		\Large
		\textbf{Universite Libre de Bruxelles} \\
		\textbf{Introduction to Language Theory and Compiling}
		
		\vspace{0.8cm}
		
		\large
		\textbf{Submitted: \today}
		
		\vfill
	\end{titlepage}
	
	% Abstract
	\begin{abstract}
		This project involves designing a compiler for the Genial Imperative Language for Learning and the Enlightenment of Students (GILLES). The grammar of the language is defined, with reserved keywords, program names, variable names, and numerical constants specified through lexical rules. Program names start with uppercase letters, while variable names begin with lowercase letters, both being case-sensitive. Integral numerical constants consist solely of digits. GILLES supports two types of comments: short comments starting with the dollar sign (\$) and long comments enclosed by double exclamation marks (!!). These comments are ignored by the scanner and are not transmitted to the parser.
	\end{abstract}

	
	% Table of Contents
	\tableofcontents
	\newpage
	
	% List of Figures (if applicable)
	\listoffigures
	\newpage
	
	% List of Tables (if applicable)
	\listoftables
	\newpage
	
	% Introduction Chapter
	\chapter{Introduction}
	\section{Background}
	Provide the background of the report here.
	\section{Objective}
	State the main objectives of your work.
	\section{Structure of the Report}
	Briefly outline the structure of your report.
	
	% Main Body Chapters
	\chapter{Literature Review}
	Discuss relevant research and literature related to your topic.
	
	\chapter{Methodology}
	Explain the methods and tools you used in your research or project.
	
	\chapter{Results}
	Present the results or findings from your work.
	
	\chapter{Discussion}
	Interpret the results and compare them with previous work.
	
	\chapter{Conclusion}
	Summarize the key points of your report and discuss possible future work.
	
	% Bibliography/References
	\chapter*{References}
	\addcontentsline{toc}{chapter}{References}
	\bibliographystyle{plain}
	\bibliography{references}  % You can also enter references manually here
	
	% Appendix (if applicable)
	\appendix
	\chapter{Appendix A}
	Include supplementary material like raw data or additional figures.
	
\end{document}
