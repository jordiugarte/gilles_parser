\documentclass[12pt,a4paper]{report}
\usepackage{verbatim}
\usepackage{graphicx}       % For including images
\usepackage{amsmath}        % For math symbols
\usepackage{amssymb}        % For more math symbols
\usepackage{hyperref}       % For hyperlinks in the document
\usepackage{geometry}       % For adjusting margins
\geometry{left=3cm,right=3cm,top=2.5cm,bottom=2.5cm}
\usepackage{fancyhdr}       % For custom headers and footers
\usepackage{setspace}       % For line spacing
\usepackage{forest}

% Custom header and footer
\pagestyle{fancy}
\fancyhf{}
\fancyhead[L]{Lexer Compiler}
\fancyhead[R]{\thepage}

\begin{document}
	
	% Title Page
	\begin{titlepage}
		\centering
		\vspace*{2cm}
		
		\Huge
		\textbf{Lexer Compiler}
		
		\vspace{1.5cm}
		
		\Large
		\textbf{Siddharth SAHAY} \\
		\textbf{Jordi UGARTE}		
		
		\vfill
		
		\Large
		\textbf{Université Libre de Bruxelles} \\
		\textbf{Introduction to Language Theory and Compiling}
		
		\vspace{0.8cm}
		
		\large
		\textbf{Submitted: \today}
		
		\vfill
	\end{titlepage}
	
	% Abstract
	\begin{abstract}
		This project involves designing and developing a compiler for the language Professor Gilles GEERAERTS implied for the instructions for this project. This language is called the Genial Imperative Language for Learning and the Enlightenment of Students (GILLES). Its grammar is stated in Table 1.\\ It is defined with reserved keywords, program names, variable names, and numerical constants specified through lexical rules. Program names start with uppercase letters, while variable names begin with lowercase letters, both being case-sensitive. Integral numerical constants consist solely of digits. GILLES supports two types of comments: short comments starting with the dollar sign (\$) and long comments enclosed by double exclamation marks (!!). These comments are ignored by the scanner and are not transmitted to the parser.
	\end{abstract}


	% Table of Contents
	\tableofcontents
	\newpage
	
	
	% Introduction Chapter
	\chapter{Introduction}
	\section{Background}
	According to **Aho, Lam, Sethi, and Ullman (2007)** in *Compilers: Principles, Techniques, and Tools*, the process of compiling involves several critical stages, including lexical analysis, parsing, semantic analysis, and optimization before machine code is generated. \\
	Lexing, on the other hand, is the process of tokenization to make a text be converted into lexical tokens belonging to categories defined by a "lexer" program. \\
	The project is conformed by the following files and classes:\\

	The following java files: \texttt{LexicalUnit.java}, \texttt{Main.java} and  \texttt{Symbol.java} were provided from the beggining as support for the assignment. However, the  \texttt{Main.java} file was developed later to run the lexical analyzer class to perform the tests. The project structure can be best described in Table 1.\\
	
	The source code is located in the  \texttt{src} folder, where the java files will me compiled into classes by running the following command:
	\begin{verbatim}
		$ make
	\end{verbatim}
	The previous command will also generate a .jar file called part1.jar to be runnable inside the dist folder. This jar file will run all tests inside the test folder. This can be runned by the command:
	\begin{verbatim}
		$ make test
	\end{verbatim}
	
	\forestset{
		folder/.style={align=center, font=\ttfamily, for tree={grow'=0, parent anchor=east, child anchor=west, anchor=mid west}},
		file/.style={font=\ttfamily},
	}

	\begin{forest}
		for tree={
			folder,
			edge={draw=gray},
			l sep+=10pt,
		}
		[gilles{\_}parser
		[src
		[LexicalUnit.java, file]
		[Main.java, file]
		[Symbol.java, file]
		]
		[test
		[Euclid.gls, file]
		]
		[more
		]
		[doc
		[latex
		[document.tex, file]
		]
		]
		[dist
		]
		[README.md, file]
		[Makefile, file]
		]
	\end{forest}

	\section{Objective}
	Assignment for Part 1: Produce a lexical analyzer of the compiler using JFlex.\\
	
	% Main Body Chapters
	\chapter{Part 1}
	The lexer was developed by the following classes:
	
	
	%\chapter{Part 2}
	%Explain the methods and tools you used in your research or project.
	
	%\chapter{Part 3}
	%Explain the methods and tools you used in your research or project.
	
	\chapter{Results}
	Present the results or findings from your work.
	
	\chapter{Conclusion}
	Summarize the key points of your report and discuss possible future work.
	
	% Bibliography/References
	\chapter*{References}
	\addcontentsline{toc}{chapter}{References}
	\bibliographystyle{plain}
	\bibliography{references}  % You can also enter references manually here
	
\end{document}
