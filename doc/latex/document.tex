\documentclass[12pt,a4paper]{report}
\usepackage{graphicx}       % For including images
\usepackage{amsmath}        % For math symbols
\usepackage{amssymb}        % For more math symbols
\usepackage{hyperref}       % For hyperlinks in the document
\usepackage{geometry}       % For adjusting margins
\geometry{left=3cm,right=3cm,top=2.5cm,bottom=2.5cm}
\usepackage{fancyhdr}       % For custom headers and footers
\usepackage{setspace}       % For line spacing

% Custom header and footer
\pagestyle{fancy}
\fancyhf{}
\fancyhead[L]{University Report}
\fancyhead[R]{\thepage}

\begin{document}
	
	% Title Page
	\begin{titlepage}
		\centering
		\vspace*{2cm}
		
		\Huge
		\textbf{Lexer Compiler}
		
		\vspace{1.5cm}
		
		\Large
		\textbf{Siddharth SAHAY} \\
		\textbf{Jordi UGARTE}		
		
		\vfill
		
		\Large
		\textbf{Université Libre de Bruxelles} \\
		\textbf{Introduction to Language Theory and Compiling}
		
		\vspace{0.8cm}
		
		\large
		\textbf{Submitted: \today}
		
		\vfill
	\end{titlepage}
	
	% Abstract
	\begin{abstract}
		This project involves designing and developing a compiler for the language Professor Gilles GEERAERTS implied for the instructions for this project. This language is called the Genial Imperative Language for Learning and the Enlightenment of Students (GILLES). Its grammar is stated in Table 1.\\ It is defined with reserved keywords, program names, variable names, and numerical constants specified through lexical rules. Program names start with uppercase letters, while variable names begin with lowercase letters, both being case-sensitive. Integral numerical constants consist solely of digits. GILLES supports two types of comments: short comments starting with the dollar sign (\$) and long comments enclosed by double exclamation marks (!!). These comments are ignored by the scanner and are not transmitted to the parser.
		
		\begin{table}[h]
			\centering
			\begin{tabular}{|c|c|}
			\hline
			Column 1 & Column 2 \\
			\hline
			Data 1 & Data 2 \\
			\hline
			Data 3 & Data 4 \\
			\hline\begin{table}[h]
			\centering
			\begin{tabular}{|c|c|}
				\hline
				Column 1 & Column 2 \\
				\hline
				Data 1 & Data 2 \\
				\hline
				Data 3 & Data 4 \\
				\hline
	\end{abstract}

	
	% Table of Contents
	\tableofcontents
	\newpage
	
	% List of Tables (if applicable)
	\listoftables
	\newpage
	
	% Introduction Chapter
	\chapter{Introduction}
	\section{Background}
	According to **Aho, Lam, Sethi, and Ullman (2007)** in *Compilers: Principles, Techniques, and Tools*, the process of compiling involves several critical stages, including lexical analysis, parsing, semantic analysis, and optimization before machine code is generated. These steps allow for the efficient translation of high-level language code into low-level machine instructions, ensuring that the compiled programs are fast and optimized for their specific runtime environment.
	
	This book is often referred to as the "Dragon Book" due to the dragon illustration on its cover and is a foundational text for anyone studying or working with compilers.
	
	\section{Objective}
		Assignment for Part 1: Produce a lexical analyzer of the compiler using JFlex.\\
	
	% Main Body Chapters
	\chapter{Part 1}
	Discuss relevant research and literature related to your topic.
	
	%\chapter{Part 2}
	%Explain the methods and tools you used in your research or project.
	
	%\chapter{Part 3}
	%Explain the methods and tools you used in your research or project.
	
	\chapter{Results}
	Present the results or findings from your work.
	
	\chapter{Conclusion}
	Summarize the key points of your report and discuss possible future work.
	
	% Bibliography/References
	\chapter*{References}
	\addcontentsline{toc}{chapter}{References}
	\bibliographystyle{plain}
	\bibliography{references}  % You can also enter references manually here
	
\end{document}
